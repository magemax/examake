\begin{exo}[De la représentation graphique au tableau de valeurs]

    La courbe ci-dessous est la représentation de la fonction f dans un repère orthonormé 

    \begin{center}
    \begin{tikzpicture}
        \begin{axis}[
        axis lines=middle,
        grid=major,
        xmin=-7.1,
        xmax=7.1,
        ymin=-2.1,
        ymax=7.1,
        xlabel=$x$,
        ylabel=$y$,
        xtick={-7,-6,...,7},
        ytick={-7,-6,...,7},
        scale=2.2,
        transform shape,
        ticklabel style={
                    fill=white
                },
        tick style={very thick},
        axis equal image,
        legend style={
        at={(rel axis cs:0,1)},
        anchor=north west,draw=none,inner sep=0pt,fill=gray!10}
        ]
        \addplot[line width=0.5mm, color=red] coordinates {
            {w_coords}
            };
        \end{axis}
    \end{tikzpicture}
    \end{center}
    
    A. Complétez le tableau de valeurs suivant correspondant au graphique :
    
    {\centering
            \renewcommand{\arraystretch}{2}
            \begin{tabular}{|c|*{4}{P{1cm}|}}
                 \hline
                 $x$& {w_xcoords} \\
                 \hline
                 $f(x)$ & {w_ycoords} \\
                 \hline
            \end{tabular}%
            
    }
    \vspace{0.5cm}
    B. Complétez chacune des phrases suivantes en fonction des informations contenues dans le graphique : 
        \begin{enumerate}
            {w_randomquestions}
        \end{enumerate}
    \end{exo}