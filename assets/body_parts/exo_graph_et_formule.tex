\begin{exo}[Graphique et formule] 
    On donne les représentations graphiques de quatre fonctions différentes A, B, C et D :

\end{exo}

\vspace{0.5cm}
\begin{tikzpicture}[scale=1]
% Repère 1
\begin{axis}[width=0.3\textwidth, axis lines=middle, xmin=-4.1, xmax=4.1, ymin=-4.5, ymax=4.1, xlabel={$x$}, ylabel={$A(x)$}, 
        grid=major,xlabel style={at={(1,0.7)}, anchor=north}, ylabel style={at={(0.5,1)}, anchor=south},
        xtick={-7,-6,...,7},
        ytick={-7,-6,...,7}]
    \addplot[blue, domain=-5:5, line width = 1.5pt,samples=100] {{w_A_latex}};
\end{axis}
\begin{axis}[xshift=4cm,width=0.3\textwidth, axis lines=middle, xmin=-4.1, xmax=4.1, ymin=-4.1, ymax=4.1, xlabel={$x$}, ylabel={$B(x)$}, 
        grid=major,xlabel style={at={(1,0.7)}, anchor=north}, ylabel style={at={(0.5,1)}, anchor=south},
        xtick={-7,-6,...,7},
        ytick={-7,-6,...,7}]
    \addplot[red, domain=-5:5,line width = 1.5pt, samples=100] {{w_B_latex}};
\end{axis}
\begin{axis}[xshift=8cm, width=0.3\textwidth, axis lines=middle, xmin=-4.1, xmax=4.1, ymin=-4.1, ymax=4.1, xlabel={$x$}, ylabel={$C(x)$}, 
        grid=major,xlabel style={at={(1,0.7)}, anchor=north}, ylabel style={at={(0.5,1)}, anchor=south},
        xtick={-7,-6,...,7},
        ytick={-7,-6,...,7}]
    \addplot[purple, domain=-5:5,line width = 1.5pt, samples=100] {{w_C_latex}};
\end{axis}
\begin{axis}[xshift=12cm, width=0.3\textwidth, axis lines=middle, xmin=-4.1, xmax=4.1, ymin=-4.1, ymax=4.1, xlabel={$x$}, ylabel={$D(x)$}, 
        grid=major,xlabel style={at={(1,0.7)}, anchor=north}, ylabel style={at={(0.5,1)}, anchor=south},
        xtick={-7,-6,...,7},
        ytick={-7,-6,...,7}]
    \addplot[black, domain=-5:5, line width = 1.5pt, samples=100] {{w_D_latex}};
\end{axis}
\end{tikzpicture}
\begin{enumerate}
    \item Parmi les "courbes" ci-dessus, laquelle pourrait représenter la fonction  $f$ définie par $f(x)={w_formule_1}$ ? Répondez ci-dessous en \textbf{expliquant votre raisonnement.}
    \item Dans cette question, on considère le progrmme de calcul suivant :
            \begin{itemize}
                {w_etapes_programme}
            \end{itemize}
        \begin{enumerate}
            \item Donnez le résultat du programme quand le nombre entré est {w_prog_nombre_entre}
            \item Sachant que l'une des courbes ci-dessus représente la fonction $g$ définie par le programme de calcul. Laquelle pourrait lui correspondre? 
            \item Donnez la formule développée et réduite de $g(x)$ en fonction de x.
        \end{enumerate}
    \item parmi les formules ci-dessous, laquelle (ou lesquelles) pourraient être celles d'une des fonctions représentées. Pourquoi ? 
        
    \[ {w_formules_3} \]
\end{enumerate}

\Texteouvide{ Corrigé :

\begin{enumerate}
    \item En testant différentes valeurs d'entrées dans $f$, on obtient $f(0)={w_f0}$ et $f({w_autreval})={w_fav}$. La droite représentant $f$ contient donc les points $(0,{w_f0})$ et $({w_autreval},{w_fav})$. La seule qui répond à ces critères est la courbe ${w_rep_1}$
    \item \begin{enumerate}
            \item On applique le programme à ${w_prog_nombre_entre}$. On obtient effectivement ${w_prog_from_a}={g_a}$
            \item En calculant comme au 1 les valeur de sorties de $g$ en plusieurs points, on obtient que la seule courbe compatible est la courbe {w_rep_2}
            \item On applique la formule à un nombre $x$, puis on développe. On obtient {w_formulecalc}
        \end{enumerate}
    \item En appliquant la formule en deux points à chaque fois, on obtient que : \begin{itemize}
        {w_reponse_4}
        \begin{itemize}
\end{enumerate}
}


