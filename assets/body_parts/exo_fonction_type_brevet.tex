\begin{exo}
    On considère le programme de calcul suivant : 
    \begin{itemize}
        \item Choisir un nombre
        {w_items_programme}
    \end{itemize}

    \begin{enumerate}
        \item \begin{enumerate} 
            \item Montrer que si le nombre de départ choisi est ${w_antecedent1}$, le nombre d'arrivée est ${w_image1}$.
            \item Qu'obtient t'on si le nombre de départ est ${w_antecedent1b}$
            \item Même question si le nombre de départ est ${w_antecedent1c}$. On exprimera la réponse sous forme de fraction irréductible.
        \end{enumerate}
        \item On appelle $h$ la fonction représentée par le programme de calcul. Montrer que $h(x)={w_formule}$
        \item Calculer, par la méthode de votre choix, un antécédent du nombre ${w_image3}$
    \end{enumerate}

\end{exo}
\vspace{0.5cm}
