\begin{exo}
    On considère le programme de calcul suivant : 
    \begin{itemize}
        \item Choisir un nombre
        {w_items_programme}
    \end{itemize}

    \begin{enumerate}
        \item \begin{enumerate} 
            \item Montrer que si le nombre de départ choisi est ${w_antecedent1}$, le nombre d'arrivée est ${w_image1}$.
            \item Qu'obtient t'on si le nombre de départ est ${w_antecedent1b}$
            \item Même question si le nombre de départ est ${w_antecedent1c}$. On exprimera la réponse sous forme de fraction irréductible.
        \end{enumerate}
        \item On appelle $h$ la fonction représentée par le programme de calcul. Montrer que $h(x)={w_formule}$
        \item Calculer, par la méthode de votre choix, un antécédent du nombre ${w_image3}$
    \end{enumerate}

\end{exo}
\vspace{0.5cm}

\Texteouvide{

On remarque d'abord que si on appelle $x$ la valeur de départ, on obtient :
\begin{itemize}
    {w_etapes_x}
\end{itemize}

\begin{enumerate}
    \item \begin{enumerate} 
        \item En remplaçant $x$ par ${w_antecedent1}$ dans la formule finale ci-dessus, on obtient bien ${w_form1}={w_image1}$
        \item On remplace $x$ par ${w_antecedent1b}$. Les calculs donnent comme valeur de sortie le nombre ${w_image1b}$
        \item On remplace $x$ par ${w_antecedent1c}$. On obtient ${w_image1c}$
    \end{enumerate}
    \item On développe la formule finale obtenue plus haut, puis on la réduit (en additionnant les deux nombres ne dépendant pas de $x$). On obtient bien {w_calculformule}.
    \item Chacune de ces deux méthodes convient (dans ce cas précis): 
    \begin{enumerate}
        \item Méthode : "On remonte le programme de calcul" : On prend les étapes en partant du résultat ${w_image3}$, et en inversant chaque étape :
        \begin{itemize}
            {w_etapes_remontada}
        \end{itemize}
        \item Méthode "On résoud l'équation comme un grand" : On part de l'équation $h(x)={w_image3}$ et on essaye d'atteindre une égalité de la forme $x=\dots$ en effectuant des opérations licites : ajouter le même nombre des deux côtés de l'égalité, ou multiplier les deux côtés de l'égalité par le même nombre non nul.
        \begin{itemize}
            {w_etapes_resolution}
        \end{itemize}
    \end{enumerate}
\end{enumerate}
}