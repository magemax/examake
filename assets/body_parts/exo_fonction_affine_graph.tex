\begin{exo}
    Sur le repère ci-dessous figure la droite $(d)$ représentant une fonction affine $f$. 

    \begin{enumerate}
        \item Par une lecture graphique, déterminer :
        \begin{enumerate}
            \item Le coefficient directeur de la droite $(d)$
            \item L'ordonnée à l'origine de la fonction
            \item La formule qui exprime $f(x)$ en fonction de $x$
        \end{enumerate}
        \item On appelle $g$ la fonction affine telle que ${w_formule_g}$.
        \begin{enumerate}
            \item Calculer $g(0)$ et $g(1)$
            \item Tracer la droite $(e)$ qui représente la fonction $g$ dans le repère ci-dessous
            \item En déduire une valeur approchée d'un nombre $a$ tel que $f(a)=g(a)$. Expliquer la méthode employée.
        \end{enumerate}
    \end{enumerate}
\end{exo}


\vspace{0.5cm}
\begin{tikzpicture}[scale=1]
% Repère 1
\begin{axis}[width=\textwidth, axis lines=middle, xmin=-8.1, xmax=8.1, ymin=-8.5, ymax=8.1, xlabel={$x$}, ylabel={$y$}, 
        grid=major,xlabel style={at={(1,0.7)}, anchor=north}, ylabel style={at={(0.5,1)}, anchor=south},
        xtick={-7,-6,...,7},
        ytick={-7,-6,...,7}]
    \addplot[blue, domain=-9:9, line width = 1.5pt,samples=100] {{w_f_latex}};
    \Texteouvide{\addplot[red, domain=-9:9, line width = 2.5pt,samples=100] {{w_g_latex}};}
\end{axis}
\end{tikzpicture}

\vspace{0.5cm}

\Texteouvide{
    \textbf{CORRIGE :}
    
    \begin{enumerate}
        \item \begin{enumerate}
            \item {w_rep1a}
            \item {w_rep1b}
            \item {w_rep1c}
        \end{enumerate}
        \item On appelle $g$ la fonction affine telle que ${w_formule_g}$.
        \begin{enumerate}
            \item {w_rep2a}
            \item {w_rep2b}
            \item {w_rep2c}
        \end{enumerate}
    \end{enumerate}
}